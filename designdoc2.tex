\documentclass{article}
\usepackage{amsthm}
\usepackage{amsmath}
\usepackage{amssymb}
\title{CS3410 Project 1 Design Document}
\date{February 25, 2015}
\author{Alexandra Anderson, Joshua Hull}
\begin{document}
\maketitle

\section*{Introduction}
Our MIPS processor will follow a five stage pipeline as discussed in class.  Each stage will be blocked off with registers storing necessary data.  Instructions will proceed through one stage of the pipeline per clock cycle.

\section*{Fetch}
This stage Fetches the next instruction from the instruction memory and updates the program counter.  Both the instruction and PC$+4$ are passed into the register so that the next stage (Decode) has access to that information.  

This stage also includes a multiplexer to select between PC$+4$ and jump/branch targets.  

\section*{Decode}
This stage uses the bits in the instruction to determine several different values:

It determines the immediate value for immediate operations (this value is computed regardless of whether or not the operation is an immediate operation - it is selected for using a multiplexer in the next stage).

It determines which registers to read from (and reads those registers).

It determines a shift amount and op-code to send to the ALU.

It also determines jump and branch amounts (though these values are not used).

The registers at the end of this stage will store certain data for the next stage including the values of the read registers, PC$+4$, jump and branch targets, the immediate value, the shift amount, the op code, the write enable bit, and the register to write to.

\section*{Execute}
This section sign-extends the immediate value and selects whether to use register B or the immediate value in the operation.  Then, it performs the given ALU operation.

This result is then combined with some additional logic (comparators in the case of slt for instance) and the result of the operation is outputted.

The registers at the end of this stage will store the result of the operation, PC$+4$, jump and branch targets, the result of the ALU operation, the write enable bit, and the register to write to.

\section*{Memory}
This section will be a pass-through stage. There will not be any logic here, though it will be blocked off by registers so all operations have to spend a cycle in this stage.

\section*{Writeback}
This section will only wright the results of the register and immediate operations that we have to implement.  It will not write the bits read from memory (as these operations will not be implemented).

\end{document}